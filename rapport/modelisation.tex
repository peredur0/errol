La dernière phase de ce projet est de réaliser un modèle d'intelligence artificielle capable de classifier les mails, en se basant sur les données extraites des phases précédentes.
Nous allons tester ici 2 types de modèle et voir si l'un est plus performant qu'un autre.

\subsection*{Généralités}
    \paragraph{Random Tree Forest}
        Une Random Tree Forest va se baser sur un ensemble d'arbre de décision pour réaliser la catégorisation.
        L'idée principale d'un arbre de décision est de sélectionner une variable dit de segmentation, qui va être utilisée pour couper l'ensemble de données en deux.
        Par défault scikit learn utilise l'indice de GINI pour déterminer cette variable.
        L'équation suivante est utilisée dans le  \emph{RandomTreeClassifier}

        \begin{eqnarray*}
            H(Q_m) = \sum_k p_{mk} (1 - p_{mk})
        \end{eqnarray*}

        $Q_m$ représente le nœud dans l'arbre de décision.
        $p_{mk}$ est la proportion des éléments de la classe $k$ dans le nœud.
        Le résultat de ce calcul va donner une idée de la dispersion des données au sein du groupe.
        Un indice Gini à 0 pour un nœud signifie que tous les éléments de ce nœud sont du même type.
        Un indice de $0.5$, correspond au niveau maximum de distribution.
        Il y a autant d'éléments de chaque catégorie dans le groupe observé.\\

        Pour chaque nœud l'algorithme va alors chercher la meilleure caractéristique pour séparer les données.
        Cette opération va se répéter jusqu'à arriver à un nœud pur ou à la profondeur maximale autorisée.\\

        Les arbres sont créés en utilisant des échantillons aléatoires des données.
        La classification finale est décidé au résultat majoritaire entre tous les arbres.\\

        Les paramètres principaux sont :
        \begin{itemize}
            \item n\_estimator - nombre d'arbres - ici, on utilise la valeur par défaut $100$
            \item max\_depth - profondeur maximum de l'arbre - en se basant sur la distribution des mails dans les dimensions du vecteur (voir Figure~\ref{fig:tfidf_distr}), on va aller à une profondeur de $100$
        \end{itemize}

    \paragraph{Support Vector Machine}
        L'algorithme Support Vector Machine va chercher à trouver un hyperplan capable de séparer les données.
        Cet hyperplan doit avoir la plus grande marge possible pour être capable de généraliser le plus possible.

        Plusieurs variables sont paramétrables pour un modèle SVM\@.
        L'utilisation de \emph{GridSearchCV} va permettre de fournir une liste de valeur possible pour chaque paramètre et de sélectionner association la plus performante.
        Les paramètres donnés à \emph{GridSearchCV} pour évaluation sont :
        \subparagraph{Kernel}
            Le Kernel est fonction noyau utilisée pour transformer les données et trouver un hyperplan linéaire.
            Dans notre cas, on n'utilisera que le noyau \emph{rbf} (Radial Basis Function).
            Ce mode semble être plus adapté à nos données qu'un noyau linéaire ou polynomial.

        \subparagraph{C}
            Ce paramètre va permettre de donner une tolérance à l'erreur plus grande lors de phase d'apprentissage.
            Cela va permettre d'avoir des marges plus grandes autour de l'hyperplan, et d'avoir une meilleure capacité de généralisation.
            Cette opération se fait en attribuant un certain poids aux erreurs.
            Plus le paramètre C va être petit plus les marges seront grandes.\\
            Valeurs utilisées - $[0.1, 1, 5, 10, 50, 100]$

        \subparagraph{Gamma}
            Ce paramètre n'est utilisé qu'avec les noyaux non linéaires.
            Il permet de gérer la distance pour que 2 points soient considérés comme faisant partie d'un même groupement.
            Avec une valeur élevée de Gamma, il faut que ces 2 points soient très proche.
            A contrario, une valeur faible les observations seront réparties dans des groupes quasi linéaires.\\
            Valeurs utilisées - $[0.0001, 0.001, 0.005, 0.01, 1, 10]$

    \paragraph{Normalisation}
        La normalisation doit permettre de mettre à la même échelle toutes les caractéristiques d'un jeu de données.
        Nous avons dans notre dataset une grande disparité d'échelle et de distribution.
        La normalisation Standard, disponible dans scikit learn, va transformer chaque caractéristique pour que la moyenne soit égale à 0 avec un écart type de 1.
        Cette opération doit permettre de faciliter le traitement par les algorithmes d'apprentissage automatique.


\subsection{Création des modèles}
    Cette section va s'intéresser aux étapes de la création des modèles.

    \paragraph{Phase préparatoire}
        L'ensemble des données (vecteur tfidf et caractéristiques retenues) est récupérée depuis la base Postgres et stockée dans un dataframe Pandas.
        L'ensemble des caractéristiques du dataset est normalisé pour harmoniser les échelles.\\

        Trois sous-ensembles seront utilisés :
        \begin{itemize}
            \item Vecteur TFIDF + caractéristiques (nombre, url, nombre\_hapax, ratio\_hapax\_uniques, majuscules, espaces)
            \item Vecteur TFIDF
            \item Id\_categorie - correspond à la table de résultat
        \end{itemize}
        Ces jeux de données sont séparés en ensemble d'entrainement ($75\%$) et de test ($25\%$) aléatoirement.

        L'identifiant de la catégorie \emph{ham} est également récupéré pour être utilisé comme résultat positif.
        
    \paragraph{Entrainement}
        Les deux types de modèles Random Tree Forest (rtf) et Support Vector Machine (svm) sont entrainé avec les ensembles vecteur seul et avec addons.
        \begin{lstlisting}[title=Entrainement des modèles, language=Python,label={lst:train}]
alg_decision_tree = RandomForestClassifier(n_estimators=100, max_depth=100, n_jobs=-1)
 model = alg_decision_tree.fit(x_train, y_train)
 ...
alg_svm = svm.SVC()
svm_params = {'kernel': ['rbf'],
              'gamma': [0.0001, 0.001, 0.005, 0.01, 1, 10],
              'C': [0.1, 1, 5, 10, 50, 100]}
hyper_params_grid = GridSearchCV(alg_svm, svm_params, cv=2, scoring='accuracy', n_jobs=-1)
hyper_params_models = hyper_params_grid.fit(x_train, y_train)
best_svm = hyper_params_models.best_estimator_
        \end{lstlisting}

        La recherche du modèle SVM nécessite l'utilisation de GridSearchCV pour trouver la meilleure combinaison entre les différents paramètres possibles.


    \paragraph{Evaluation des modèles}
        L'évaluation des modèles générés est faites avec les données de test mises de cotés lors de la phase de préparation.
        \begin{lstlisting}[title=Evaluation des modèles, language=Python,label={lst:train}]
predictions = model.predict(x_test)
precision, recall, fscore, _ = score(y_test, predictions, pos_label=pos_label, average='binary')
...
predictions = best_svm.predict(x_test)
precision, recall, fscore, _ = score(y_test, predictions, pos_label=pos_label, average='binary')
        \end{lstlisting}
        Les métriques utilisées sont :
        \begin{itemize}
            \item Precision (P) - ici Ham correctement classifiés
            \item Recall (R) - ici Spam correctement classifiés
            \item Accuracy - Documents correctement classifiés sur l'ensemble du jeu de test
            \item Fscore - Moyenne harmonique utilisée dans les classes sont déséquilibrées - $2(P \cdot R)/(P + R)$
        \end{itemize}

        A l'issue de cette phase d'évaluation, les modèles sont sauvegardés en utilisant le module pickle.

    \subsubsection{Résultat}
\begin{verbatimtab}
$ ./errol.py train rtf svm
22:32:24,888 : Initialisation du programme pour la phase train - OK
22:32:24,888 : Entrainement des modèles
22:32:24,888 : Récupération des caractéristiques
22:32:26,068 : Nombre de messages: 5333 - Nombre de caractéristiques: 2951
22:32:28,439 : ID pour les ham - 1
22:32:28,439 : RandomTree Forest avec les vecteurs tfidf et caractéristiques
22:32:28,868 : RandomTree Forest avec les vecteurs tfidf seulement
22:32:29,314 : Support Vector Machine avec les vecteurs tfidf et caractéristiques
22:38:44,097 : Support Vector Machine avec les vecteurs tfidf seulement
22:45:06,374 : rtf_vect_feat - Prec: 0.981 - Rec: 0.996 - Acc: 0.984 - Fsc: 0.989
22:45:06,388 : rtf_vect_only - Prec: 0.981 - Rec: 0.981 - Acc: 0.973 - Fsc: 0.981
22:45:06,402 : svm_vect_feat - Prec: 0.984 - Rec: 0.991 - Acc: 0.982 - Fsc: 0.987
22:45:06,430 : svm_vect_only - Prec: 0.982 - Rec: 0.991 - Acc: 0.981 - Fsc: 0.986

$ du -h models/*
5,9M	models/rtf_vect_feat.pickle
6,6M	models/rtf_vect_only.pickle
18M	models/svm_vect_feat.pickle
19M	models/svm_vect_only.pickle
\end{verbatimtab}

\subsection*{Constatations}
    La première constatation concerne le temps d'entrainement.
    La génération des forets est quasiment instantanée même avec 100 arbres et une profondeur de 100.
    L'entrainement des modèles SVM via GridSearchCV a nécessité environ 7 minutes dans les deux cas et consommant 100\% des 12 CPU installé sur ma machine.\\

    La taille des modèles sauvegardés est également très différente.
    Les forêts sont généralement plus petite que les SVM\@.
    De plus, les modèles ayant été entrainés avec les données complètes (tfidf + addons) sont plus léger que les modèles analogues entrainés uniquement avec le vecteur tfidf.\\

    Concernant les métriques à proprement parler, les résultats peuvent varier en fonction du découpage aléatoire lors de la création des sets d'entrainement et de tests.
    Pour les 4 modèles le FScore est de 98\% et la différence se joue au dixième.
    Cet entrainement semble montrer que l'ajout des caractéristiques (addons) a une influence positive sur la classification.
    Cependant, cela n'a pas été systématiquement le cas lors de mes tests précédents.\\

    Ici, le modèle avec les meilleurs résultats est la Random Tree Forest entrainée avec toutes les données.
    Je pense qu'une validation pourrait être utile avec des mails qui n'ont pas été traité lors des étapes précédentes (NLP et autres).
